\chapter{\LaTeX2e{}の基本的な使い方}

\section{図の使い方}

図\ref{fig:青い丸}が青い丸である.

\begin{figure}[htbp]
 \centering
 \includegraphics[width=0.3\columnwidth]{blue-circle.pdf}
 \caption{青い丸}
 \label{fig:青い丸}
\end{figure}

\section{表の使い方}

買うものを表\ref{table:買うもの}に示す.

\begin{table}[htbp]
 \caption{買うもの}
 \label{table:買うもの}
 \centering
 \begin{tabular}{lrr}
  \hline
  品名   & 単価(円) & 個数 \\
  \hline
  りんご & 100        & 5    \\
  みかん & 50         & 10   \\
  \hline
 \end{tabular}
\end{table}

\section{参考文献の使い方}

\BibTeX{}を使うと,参考文献リストが自動で生成される.

LDPC符号はGallagerにより提案された誤り訂正符号である\cite{gallager}.

標本空間から実数へ写像する関数のことを,確率変数と呼ぶ
\cite[p.20~Definition~2.1]{mitzenmacher}.

\section{数式の使い方}

インラインでは$\sum_{k = 1}^n k$のように書く.

別行数式は,
\begin{equation}
 E = mc^2 \label{eq:Einstein}
\end{equation}
のように書く.「\pageref{eq:Einstein}ページの式\eqref{eq:Einstein}」のよ
うに,式を参照することができる.

複数の数式を並べるには,
\begin{gather}
 (a+b)^2 = a^2 + 2ab + b^2 \\
 (a-b)^2 = a^2 - 2ab + b^2 \\
 (a+b)^3 = a^3 + 3a^2b + 3ab^2 + b^3
\end{gather}
のように書く.

位置をそろえて数式を書くには,
\begin{align}
 \sinh^{-1} x &= \log(x + \sqrt{x^2 + 1}) \notag \\
              &= x - x^3 / 6 + 3x^5 / 40 + \dotsb
\end{align}
のようにする.

長い数式を複数行にわたって書くには,
\begin{multline}
 a + b + c + d + e + f + g + h + i + j + k \\
   + l + m + n + o + p + q + r + s + t + u + v \\
   + w + x + y + z + \alpha + \beta + \gamma + \delta
\end{multline}
のようにする.最初の行は左により,最後の行は右による.

ベクトルなどをイタリック体の太字で表したいときには,
\begin{equation}
 \bm{a}
\end{equation}
のようにする.

数式の中で普通の文字を使いたい場合は,
\begin{equation}
 \mathbb{I} \left[
             \text{$\bm{c}$が例のベクトルである}
            \right]
\end{equation}
のようにする.

行列は,
\begin{equation}
 A =
 \begin{pmatrix}
  a & b \\
  c & d
 \end{pmatrix}
\end{equation}
のように書ける.

数理計画問題は,
\begin{equation}
 \begin{aligned}
   & \text{minimize}   & &(x_1 + x_2 + \dots + x_n) \\
   & \text{subject to} & & x_i + x_j \ge 1, \; (i, j) \in E \\
   &                   & & x_i \in \{0, 1\}, \; i \in \{1, 2, \dots, n\}
 \end{aligned}
 \label{eq:IP}
\end{equation}
のように書くと,式番号が中央にくるのできれいかもしれない.

\section{定義・補題・定理・証明の使い方}

\begin{definition}
 定義を書く.
\end{definition}

\begin{lemma}
 補題を書く.
\end{lemma}

\begin{theorem}
 定理を書く.
\end{theorem}

\begin{theorem}[The title]
 定理を書く.
\end{theorem}

\begin{proof}
 証明を書く.
\end{proof}
